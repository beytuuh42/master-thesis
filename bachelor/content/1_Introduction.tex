 \chapter{Introduction}
Recently, artificial intelligence (AI) systems beat human players on professional, even world-class levels. OpenAI Five and AlphaStar were able to learn the games Dota 2 and StarCraft II, respectively, after many years of training and proved, that AI systems can even outperform experts in their respective fields. These systems architectures are complex and adjusted explicitly to playing those games on a professional level \cite{Vinyals2019, 2019arXiv191206680O}. 

Various algorithms are tested in virtual environments to see how they interact in different environments under different settings \cite{2018arXiv180300933H}. These algorithms can be transferred to the real-world, if proven safe and useful, to aide humans with different kinds of works \cite{openai2019solving, tobin2017domain}. 

In this thesis, state-of-the-art evaluation metrics for the performance of AI systems in Arcade video games are analyzed. This research aims to find out the suitability of these metrics and whether other metrics can be applied to get a better overview of the performance and behavior of the system.

To accomplish these goals, the techniques and methods of different AI systems and algorithms, that are proven to be effective in playing video games, including OpenAI Five and AlphaStar, are analyzed. Additionally, popular multiplayer video games are broken down to determine what factors makes a good player, followed by analyzing state-of-the-art evaluation metrics for Arcade games.

Finally, experiments are conducted with the knowledge gained and compared to state-of-the-art methods and techniques.