\section{Further games}
To cover more than RTS games, two shooter games are analyzed in this subsection, namely Fortnite and Counter-Strike: Global Offensive (CSGO). 

\subsubsection{Fortnite}
Fortnite: Battle Royale (Fortnite) is a Battle Royale online video game, created by Epic Games in 2017 \cite{EpicGame39:online}. Fortnite is a popular video game with a peak of over 12.3 million players concurrently, due to a special event \cite{FortniteTweet}. Though, Fortnite is a relative new game, compared to the previously mentioned ones, it has tournaments since 2018. The highest prize pool for a tournament in 2018 was \$1.9 million. In the following year a new tournament was organized by the publisher, the \textit{Fortnite World Cup Finals}. The tournament was split into two variations, namely solo (1v1) and duo (2v2). The prize pool, combined for both tournaments, was \$30.3 million for the 150 participants. The first placed in each variation got \$3 million, which exceeds the total prize pool for \nameref{sec:LoL} and \nameref{sec:sc2}.

In Fortnite a match consists of 100 players and they fight until the last man, or team, is standing. There are several games modes, including ranges from 1v1 to 4v4. It is a shooter game, in which the players are combating each other with weapons and additional items. The outstanding feature of this game is the concept of \textit{building}. By harvesting particular resources (wood, stone, metal), players are able to build structures such as, walls, floors stairs and roofs. These structures are used offensively, or defensively, in order to outmaneuver the enemy and eliminate him. The resources can be collected by either destroying objects in the game with the respective material, e.g. tree, house, car, or by eliminating a player and collect the dropped materials. At the start of a match, all players are flying in a bus over a big map. Without any weapons or items, players are scattering over the map in either calm, or lively areas, in order to collect weapons, items and building materials to eliminate opponents. Over the course of the game, a circle is enclosing a random spot on the map in specific intervals, until it is so small that no player can stand inside. Players outside of the circle are gradually losing health points, which forces every player to move inside the circle. At the end, one victorious player is left \cite{BattleRo48:online}. Following metrics can be used to determine the performance of a player:
\begin{itemize}
	\item aim and reaction time
		\begin{itemize}
			\item e.g. reflexes and speed of a player
		\end{itemize}
	\item weapon choice
		\begin{itemize}
			\item e.g. choosing strong weapons over weak weapons (indicated by a color pattern) and usage of weapons according to their intended range
		\end{itemize}
	\item item usage
		\begin{itemize}
			\item e.g. using bandages when low on HP
		\end{itemize}
	\item building
		\begin{itemize}
			\item e.g. effective building to outmaneuver the enemy, i.e., building a small house to hide inside when an enemy approaches, instead of a floor \cite{GuideBat52:online}
		\end{itemize}
\end{itemize}

\subsubsection{Counter-Strike: Global Offensive}
CSGO, is a successor of the popular Counter-Strike series. It is a team based first person shooter, that was published by Valve Corporation and Hidden Path Entertainment in August 2011. Its popularity reached its peak with 1.3 million concurrent players in April 2020 \cite{CounterS89:online}. Due to its popularity, different tournaments exist in different regions. The biggest ones are the \textit{CS:GO Major Championships} (Majors), which were introduced by Valve in 2013. Since then, the world's best players compete against each other in the tournament sponsored by Valve. In order to participate, teams have to qualify or be invited to the tournament. Until 2015, they were held one to three times a year, with a prize pool of \$250,000, but since 2016 the tournaments were held biannually and the prize pool has increased to \$1,000,000, with the first placed team earning \$500,000. The amount of participating teams also changed and has been increased from 16 to 24 \cite{CSGOMajo61:online}. 

In the competitive game mode, players are competing in two teams on a fixed set of maps. The first team reaching 16 wins out of 30 rounds, is victorious. In order to maximize the chances of winning the match, players have to purchase weapons, armor, defuse kits and manage their in-game economy. Each map consists of a Terrorists (T) and Counter-Terrorists (CT) base, where the players spawn. The Ts are targeting to plant their bomb on either the ``A''-spot or ``B''-spot, while the CTs have to defuse it. At the beginning of each round, the CTs are scattering around the spots and the Ts are preparing to get control of one spot, in order to plant the bomb. A round can end in various way, by either eliminating all opponents, by planting and explosion of the bomb (win for the Ts),  by defusing the bomb (win for the CTs), or by expiration of the round timer (win for the CTs) \cite{CounterS4:online}.

Since Fortnite and CSGO are both shooter games, they share similarities with performance metrics. The following listing shows metrics to determine players' performance, but is not limited to them:
\begin{itemize}
	\item aim and reaction time
		\begin{itemize}
			\item e.g. reflexes and speed of a player
		\end{itemize}
	\item weapon choice
		\begin{itemize}
			\item e.g. selection of weapons, depending on strength and economy
		\end{itemize}
	\item positioning/tactic
		\begin{itemize}
			\item e.g. positioning in advantageous places on one of the spots, to detect approaching enemies, while also having a safe place to retreat
		\end{itemize}
	\item utilization of grenades (high explosive, smoke, flashbang, molotov)
		\begin{itemize}
			\item e.g. to (re)gain control of a spot, using flashbangs, smokes, or both, on advantageous places to limit opposing vision
		\end{itemize}
\end{itemize}