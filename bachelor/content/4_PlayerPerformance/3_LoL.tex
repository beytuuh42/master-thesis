 \section[head={LoL}, tocentry={League of Legends}, reference={LoL}]{League of Legends} \label{sec:LoL}
League of Legends (LoL) is a Multiplayer Online Battle Arena (MOBA) online video game created by Riot Games \cite{NewToLea59:online}. LoL is the biggest PC game in the world with 8 million peak concurrent players every day in August, 2019 \cite{JoinUsOc5:online}. LoL has two major tournaments every year, the \textit{Mid-Season Invitational} and \textit{World Championship}. The organizer for both tournaments is Riot Games. Teams around the globe compete in regional tournaments to qualify for these both events. In 2019, 24 teams participated in the World Championship and 13 for the Mid-Season Invitational, with a prize pool of \$2.25 and \$1 million respectively \cite{PremierT40:online}.

LoL has commonalities with \nameref{sec:d2}, both are team-based strategy games where two teams of five heroes, or champions in LoL jargon, have to destroy a building in the opposing base. The core features are mostly the same:
\begin{itemize}
	\item most competitive game mode is a 5v5
	\begin{itemize}
		\item in LoL, the pick and ban phase differs in tournaments \cite{NewToLea66:online}
	\end{itemize}
	\item in each team, there are four \textit{laners} and one \textit{jungler}
	\item all champions are gathering resources, gold and XP, in order to level up their champions and buying items
	\item the strongest neutral monsters give additional bonuses \cite{NewToLea79:online}
\end{itemize}

\subsection{Metrics}
By presenting in-game statistics, the LoL client gives information on how players performed, weather with plain numerical values, or by graphs, as seen in Appendix \ref{app:lol_stats}. Besides the stats, that the LoL client offers for every match, the metrics for evaluating of players are similar to \nameref{sec:d2} and even \nameref{sec:sc2}:
\begin{itemize}
	\item in-game statistics
	\begin{itemize}
		\item e.g. kills, deaths, gold, XP
	\end{itemize}
	\item{using a rating system, that reflects the players skill level by accumulating points, that changes depending on victory or defeat}
	\begin{itemize}
		\item the difference to \nameref{sec:sc2} is, that the points are not visible to the players, but rather their tier and the points in their respective division
	\end{itemize}
	\item{reaction time}
	\begin{itemize}
		\item e.g. dodging life-threatening abilities, or casting them in the right moment
	\end{itemize}
\end{itemize}
Even though, the stats give plenty information, it is not clear to say how a player actually performed, because it can be misleading, e.g., a player can have a low amount of deaths, compared to his team, but this could be due to absence of the game. Therefore, a more common evaluation method has been established, reviewing replays. As seen in \ref{sec:sc2}, \nameref{ssec:alphastar} was able to learn the game by ``watching'' replays and playing against itself, which led to identifying errors in gameplay and exploiting them \cite{Vinyals2019}. Human players are doing the same, they are identifying their errors by reviewing their own replays and try to fix them. This is done in the professional competitive play \cite{TheStory90:online}, as well as regular competitive play \cite{Challeng53:online}. Following, but not limited to, metrics can be used to evaluate the performance of players, by reviewing replays:
\begin{itemize}
	\item positioning
	\begin{itemize}
		\item e.g. in team fights, champions with high defensive attributes should be on the front line, while champions high offensive attributes should be on the back line
	\end{itemize}
	\item champion mechanics
		\begin{itemize}
			\item e.g. effective utilization of the environment and champion abilities
		\end{itemize}
	\item itemization
		\begin{itemize}
			\item e.g. buying for the respective champion the best items at the right time 
		\end{itemize}
	\item objective control
		\begin{itemize}
			\item e.g. controlling the map with warding, destroying opposing buildings \cite{Challeng53:online}
		\end{itemize}
\end{itemize}

There are many more points that can be listed, for example, positioning in lane, or knowing what other players are capable off. However, naming them all would require in-depth analytical knowledge of the game, which is beyond the scope.
