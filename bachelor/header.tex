
%%%%%%%%%%%%%%%%%%%%%%%%%%%%%%%%%%%%%%%%%%%%%%%%%%%%%%%
%																					%
%	In dieser Datei werden alle Packages eingebunden, 	%
% welche f�r das Dokument n�tig sind. Desweiteren 		%
% werden die Dokumentinformationen gesetzt.						%
%																											%
%%%%%%%%%%%%%%%%%%%%%%%%%%%%%%%%%%%%%%%%%%%%%%%%%%%%%%%
%
%	Die KOMAScript Dokumentklasse "scrbook" verwenden.
%
\documentclass[pdftex, 		%
							a4paper, 		% DIN A4 verwenden
							titlepage,	% separate Titelseite
							%draft,			%	Draft-Version, keine Bilder im pdf!
							final,			% Final-Version
							oneside,		% einseitiger Druck
							11pt,				% Schriftgr��e 12pt
							DIV=calc,
							%tocbasic,
							headings=optiontoheadandtoc,
							listof=entryprefix,
							]{scrbook}	%	KOMAScript scrbook-Dokumentklasse
							
%%%%%%%%%%%%%%%%%%%%%%%%%%%%%%%%%%%%%%%%%%%%%%%%%%%%%%%%
%	Einbinden der Pakete 
%%%%%%%%%%%%%%%%%%%%%%%%%%%%%%%%%%%%%%%%%%%%%%%%%%%%%%%%

% PDF Dateien einbinden
\usepackage{pdfpages}

%Settings for PDF Pages to accept additonal versioned PDF files
\pdfminorversion=6
\pdfcompresslevel=9
\pdfobjcompresslevel=9

%Infos dazu unter: http://www.bakoma-tex.com/doc/latex/koma-script/scrhack.pdf
%Einige Pakete haben Probleme mit dem Komaskript.
\usepackage{scrhack} 


% Definieren von eigenen benannten Farben.
% F�r sp�tere Verwendung in dem Dokument, definieren wir einzelne
% benannte Farben.
%
\usepackage{xcolor}
\definecolor{gray1}{gray}{0.92}
\definecolor{darkgreen}{rgb}{0,0.5,0}

\definecolor{urlLinkColor}{rgb}{0,0,0.5}
\definecolor{LinkColor}{rgb}{0,0,0}
\definecolor{ListingBackground}{rgb}{0.85,0.85,0.85}


\rmfamily
%\usepackage{natbib}
\usepackage{tabularx}
\usepackage{palatino} 				% Schriftfamilie Palatino
\usepackage[ansinew]{inputenc} % Umlaute  
%\usepackage[dvips]{color}    	% f�r graue Boxen
\usepackage[dvips]{graphicx} 	% Grafikpaket
\usepackage{siunitx}
\usepackage{subfig}
\usepackage{float}
\usepackage{makeidx}   				% Paket zur Erzeugung eines Index
\usepackage[normalem]{ulem}   % bietet Unterstreichungsvarianten
\usepackage{picins} 					% Bilder im Absatz platzieren
\usepackage[T1]{fontenc}			% Erweiterten Zeichensatz aktivieren
\usepackage{multido}					% erm�glicht Schleifenartiges wiederholen von Befehlen
\usepackage{mdwlist}					% erm�glicht das Setzen des Z�hlers bei Aufz�hlungspunkten
\usepackage{paralist}					% Paket f�r Aufz�hlungen, erweitert Enumerate-Paket
\usepackage{longtable}				% mehrseitige Tabellen
\usepackage{colortbl} % f�r frabe bei table
\usepackage{tocbasic}
\parindent0pt           			% verzichte auf Einr�cken der ersten Zeile
\parskip1ex             			% Abstand zwischen den Abs�tzen

\usepackage{setspace}					% Paket zum Einstellen des Zeilenabstands
\onehalfspacing								% anderthalbfacher Zeilenabstand
%\doublespacing								% doppelter Zeilenabstand
%\singlespacing								% einfacher Zeilenabstand

%\usepackage[german]{babel}
\usepackage[american]{babel}
\usepackage{csquotes} 
\usepackage[subfigure]{tocloft}

\newcommand{\listequationsname}{List of Equations}
\newlistof{myequations}{equ}{\listequationsname}
\newcommand{\myequations}[1]{%
\addcontentsline{equ}{myequations}{\protect\numberline{\theequation}#1}\par}
\setlength{\cftmyequationsnumwidth}{2.5em}% Width of equation number in List of Equations
\setlength\cftmyequationsindent{1.5em}

%\usepackage[german=quotes]{csquotes} %Deutsche Anf�hrungszeichen

\usepackage{color}
\definecolor{LinkColor}{rgb}{0.1,0.1,0.1}
%\definecolor{ListingBackground}{rgb}{0.85,0.85,0.85}
\definecolor{ListingBackground}{rgb}{0.98,0.98,0.98}
\definecolor{gray}{rgb}{0.4,0.4,0.4}
\definecolor{darkblue}{rgb}{0.0,0.0,0.6}
\definecolor{cyan}{rgb}{0.0,0.6,0.6}


%
% Farbeinstellungen f�r die Links im PDF Dokument.
%
\makeindex

%-----------Paket f�r absolute Positionierung von Grafiken------------------
\usepackage[absolute]{textpos}
\setlength{\TPHorizModule}{1mm}
\setlength{\TPVertModule}{\TPHorizModule}

%-----------Aufz�hlungen und Einstellungen f�r Sourcecode-------------------
%\usepackage[savemem]{listings} %Bei wenig Arbeitsspeicher dies Option [savemem] aktivieren.
\usepackage{listings}
\lstloadlanguages{TeX,XML, Java} % TeX sprache laden, notwendig wegen option 'savemem'
\lstset{%
	language=[LaTeX]TeX,     % Sprache des Quellcodes ist TeX
	numbers=left,            % Zelennummern links
	stepnumber=1,            % Jede Zeile nummerieren.
	numbersep=5pt,           % 5pt Abstand zum Quellcode
	numberstyle=\tiny,       % Zeichengr�sse 'tiny' f�r die Nummern.
	breaklines=true,         % Zeilen umbrechen wenn notwendig.
	breakautoindent=true,    % Nach dem Zeilenumbruch Zeile einr�cken.
	postbreak=\space,        % Bei Leerzeichen umbrechen.
	tabsize=2,               % Tabulatorgr�sse 2
	basicstyle=\ttfamily\footnotesize, % Nichtproportionale Schrift, klein f�r den Quellcode
	showspaces=false,        % Leerzeichen nicht anzeigen.
	showstringspaces=false,  % Leerzeichen auch in Strings ('') nicht anzeigen.
	extendedchars=true,      % Alle Zeichen vom Latin1 Zeichensatz anzeigen.
	backgroundcolor=\color{ListingBackground}} % Hintergrundfarbe des Quellcodes setzen.


\lstset{
  basicstyle=\small\ttfamily,
  columns=fullflexible,
  showstringspaces=false,
  %commentstyle=\color{gray}\upshape
}
%neue Lang definieren, als Bsp.
\lstdefinelanguage{XML-changed}
{
  basicstyle=\footnotesize\ttfamily\bfseries,
  morestring=[b]",
  morestring=[s]{>}{<},
  morecomment=[s]{<?}{?>},
  stringstyle=\color{black},
  identifierstyle=\color{darkblue},
  keywordstyle=\color{cyan},
  morekeywords={xmlns,version,type}% list your attributes here
}

%-----------Caption Package-------------------
\usepackage{caption}
\DeclareCaptionFont{white}{\color{white}}
\DeclareCaptionFormat{listing}{\colorbox[cmyk]{0.43, 0.35, 0.35,0.01}{\parbox{\textwidth}{\hspace{15pt}#1#2#3}}}

\DeclareCaptionFormat{graphics}{\colorbox[cmyk]{0.43, 0.35, 0.35,0.01}{\parbox{\textwidth}{\hspace{15pt}#1#2#3}}}

%-----------Header+Footer---------------------------------------------------
\usepackage{fancyhdr}					%
\pagestyle{fancy}							%

\fancyhead{}
\fancyfoot{} 
\renewcommand{\headrulewidth}{0.4pt} % Kopzeilenlinie
\renewcommand{\footrulewidth}{0.0pt} % Fusszeilenlinie 0.0pt blendet sie aus

\renewcommand{\chaptermark}[1]{\markboth{\thechapter\quad#1}{}}
\renewcommand{\sectionmark}[1]{\markright{\thesection\quad#1}}

\fancyhead[LO]{\small\sffamily\rightmark}
\fancyhead[RO]{\small\sffamily\thepage}

%-----------Um die Eidesstattliche Erkl�rung als PDF einzubinden-----------------------------------
%\usepackage{pdfpages}

%------------Glossar--------------------------------------------------------------
\usepackage{expdlist}
\usepackage{glossar}

\renewcommand{\glshead}{\chapter*{Glossar}}
\renewcommand{\glentry}[2]{\glossary{#1@[#1] #2|glspage}}
\renewcommand{\glsgroup}[1]{{\listpart{\makebox[0pt][l]{\rule[-2pt]{\textwidth}{0.5pt}}{\textbf{\large #1}}}}}

\makeglossary

%Glossar mit Bordmitteln ------------------------------------------------------------------
%Darstellung des Glossars einstellen
%\usepackage[style=super, header=none, border=none, number=none, cols=2, toc=true]{glossary}
%\renewcommand{\glossaryname}{Glossar}
%\printglossary

% --- diverse Schriften -------------------------------------------------------------------
\newcommand{\url}[1]{{\sf\small #1}}     % Hyperlinks


% -------F�r ToDo-Notes--------------------------------------------------------------------
\usepackage[color=red, shadow]{todonotes} % ", disable" deaktiviert ToDo-Notes
%Vereinfachtes "Inline-Todo"
\newcommand{\td}[1]{{\todo[inline]{#1}}}
\newcommand{\tdu}[1]{{\todo[inline, color=green!40]{#1}}}

%--------F�r Links-------------------------------------------------------------------------


%%%%%% Hinzuf�gen einer 4. Ebene als paragraph

%\makeatletter
%\renewcommand\paragraph{\@startsection{paragraph}{4}{\z@}%
      %      {-2.5ex\@plus -1ex \@minus -.25ex}%
   %         {1.25ex \@plus .25ex}%
 %           {\normalfont\normalsize\bfseries}}
%\makeatother
%\setcounter{secnumdepth}{4} % how many sectioning levels to assign numbers to
%\setcounter{tocdepth}{4} 


%--------HyperRef konfigurieren-------------------------------------------------------------------------

\usepackage[
	pdftitle={\TitelArbeit},
	pdfauthor={\DeinName},
	pdfsubject={\TitelArbeit},
	pdfcreator={MiKTeX, LaTeX with hyperref and KOMA-Script auf Basis der Vorlage von seiler.it},
	pdfkeywords={Abschlussarbeit, Stuttgart, Hochschule, Hochschule der Medien, K�nstliche Intelligenz, openai},%weitere Keywords hier einf�gen
	pdfpagemode=UseOutlines,%                                  
	pdfdisplaydoctitle=true,%                                  
	pdflang=de%                                              
]{hyperref}
\usepackage{apacite}
\hypersetup{%
	colorlinks=true,%        Aktivieren von farbigen Links im Dokument (keine Rahmen)
	linkcolor=LinkColor,%    Farbe festlegen.
	citecolor=LinkColor,%    Farbe festlegen.
	filecolor=LinkColor,%    Farbe festlegen.
	menucolor=LinkColor,%    Farbe festlegen.
	urlcolor=LinkColor,%     Farbe von URL's im Dokument.
	bookmarksnumbered=true%  �berschriftsnummerierung im PDF Inhalt anzeigen.
}


\captionsetup[table]{
	format=plain,
  labelsep = newline,
  textfont = {bf,it}, 
  justification=centering,
	singlelinecheck=false,
	skip = \medskipamount
	}
	
	\captionsetup[figure]{
	format=plain,
	labelfont = it,
  justification=justified,
	singlelinecheck=false,
	skip = \medskipamount
	}
	
\usepackage[acronym, nonumberlist, nopostdot]{glossaries}

\makeglossaries

\newlength{\acronymlabelwidth}
\setlength{\acronymlabelwidth}{0.25\textwidth}
\newglossarystyle{listwithwidth}{%
  \renewenvironment{theglossary}%
    {\begin{description}}{\end{description}}%
  \renewcommand*{\glossaryheader}{}%
  \renewcommand*{\glsgroupheading}[1]{}%
  \renewcommand*{\glossaryentryfield}[5]{%
    \item[\parbox{\acronymlabelwidth}{\glsentryitem{##1}\glstarget{##1}{##2}}]
       ##3\glspostdescription\space ##5}%
  \renewcommand*{\glossarysubentryfield}[6]{%
    \glssubentryitem{##2}%
    \glstarget{##2}{\strut}##4\glspostdescription\space ##6.}%
  \renewcommand*{\glsgroupskip}{}
}

\newacronym{ai}{AI}{Artificial Intelligence}

\newacronym{rl}{RL}{Reinforcement Learning}

\newacronym{ea}{EA}{Evolutionary Algorithms}

\newacronym{ml}{ML}{Machine Learning}

\newacronym{mdp}{MDP}{Markov Decision Process}

\newacronym{dqn}{DQN}{Deep Q-Learning}

\newacronym{dnn}{DNN}{Deep Artificial Neural Network}

\newacronym{nn}{NN}{Neural Network}

\newacronym{ann}{ANN}{Artificial Neural Network}

\newacronym{cnn}{CNN}{Deep Convolutional Artificial Neural Network}

\newacronym{td}{TD}{Temporal Difference}

\newacronym{a2c}{A2C}{Advantage Actor Critic}

\newacronym{ga}{GA}{Genetic Algorithms}

\newacronym{es}{ES}{Evolution Strategies}

\newacronym{sc2}{SC2}{StarCraft II}

\newacronym{rts}{RTS}{Real-time Strategy}

\newacronym{1v1}{1V1}{One versus One}

\newacronym{sc2le}{SC2LE}{StarCraft II Learning Environment}

\newacronym{mmr}{MMR}{Matchmaking Rating}

\newacronym{apm}{APM}{Action Per Minute}

\newacronym{fps}{FPS}{Frames Per Second}

\newacronym{xp}{XP}{Experience Points}

\newacronym{hp}{HP}{Health Points}

\newacronym{lstm}{LSTM}{Long short-term memory}

\newacronym{ppo}{PPO}{Proximal Policy Optimization}

\newacronym{moba}{MOBA}{Multiplayer Online Battle Arena}

\newacronym{csgo}{CSGO}{Counter-Strike: Global Offensive}

\newacronym{ct}{CT}{Counter-Terrorists}

\newacronym{t}{T}{Terrorists}
%\newglossaryentry{setS}
{ 
	name={$\mathcal{S}$},
	description={finite set of states}
}

\newglossaryentry{setA}
{ 
	name={$\mathcal{A}$},
	description={finite set of actions}
}

\newglossaryentry{rewR}
{ 
	name={$\mathcal{R}$},
	description={reward function}
}

\newglossaryentry{gamma}
{ 
	name={$\gamma$},
	description={discount factor}
}

\newglossaryentry{pbmatrix}
{ 
	name={$\mathcal{P}$},
	description={state transition probability matrix}
}

\newglossaryentry{gt}
{ 
	name={$\mathcal{G_{t}}$},
	description={return - discounted sum of immediate rewards}
}

\newglossaryentry{pi}
{ 
	name={$\pi$},
	description={policy}
}

\newglossaryentry{setS}
{ 
	name={$\mathcal{S}$},
	description={a set of states}
}

\newglossaryentry{setS}
{ 
	name={$\mathcal{S}$},
	description={a set of states}
}

\newglossaryentry{setS}
{ 
	name={$\mathcal{S}$},
	description={a set of states}
}

\newglossaryentry{setS}
{ 
	name={$\mathcal{S}$},
	description={a set of states}
}

\newglossaryentry{setS}
{ 
	name={$\mathcal{S}$},
	description={a set of states}
}

\newglossaryentry{setS}
{ 
	name={$\mathcal{S}$},
	description={a set of states}
}

\newglossaryentry{setS}
{ 
	name={$\mathcal{S}$},
	description={a set of states}
}

\newglossaryentry{setS}
{ 
	name={$\mathcal{S}$},
	description={a set of states}
}

\newglossaryentry{setS}
{ 
	name={$\mathcal{S}$},
	description={a set of states}
}

\newglossaryentry{setS}
{ 
	name={$\mathcal{S}$},
	description={a set of states}
}

\newglossaryentry{setS}
{ 
	name={$\mathcal{S}$},
	description={a set of states}
}

\newglossaryentry{setS}
{ 
	name={$\mathcal{S}$},
	description={a set of states}
}

\newglossaryentry{setS}
{ 
	name={$\mathcal{S}$},
	description={a set of states}
}

\newglossaryentry{setS}
{ 
	name={$\mathcal{S}$},
	description={a set of states}
}

\newglossaryentry{setS}
{ 
	name={$\mathcal{S}$},
	description={a set of states}
}

\newglossaryentry{setS}
{ 
	name={$\mathcal{S}$},
	description={a set of states}
}

\newglossaryentry{setS}
{ 
	name={$\mathcal{S}$},
	description={a set of states}
}

\newglossaryentry{setS}
{ 
	name={$\mathcal{S}$},
	description={a set of states}
}

\newglossaryentry{setS}
{ 
	name={$\mathcal{S}$},
	description={a set of states}
}

\newglossaryentry{setS}
{ 
	name={$\mathcal{S}$},
	description={a set of states}
}

\newglossaryentry{setS}
{ 
	name={$\mathcal{S}$},
	description={a set of states}
}

\newglossaryentry{setS}
{ 
	name={$\mathcal{S}$},
	description={a set of states}
}

\newglossaryentry{setS}
{ 
	name={$\mathcal{S}$},
	description={a set of states}
}

\newglossaryentry{setS}
{ 
	name={$\mathcal{S}$},
	description={a set of states}
}

\newglossaryentry{setS}
{ 
	name={$\mathcal{S}$},
	description={a set of states}
}

\newglossaryentry{setS}
{ 
	name={$\mathcal{S}$},
	description={a set of states}
}

\newglossaryentry{setS}
{ 
	name={$\mathcal{S}$},
	description={a set of states}
}

\newglossaryentry{setS}
{ 
	name={$\mathcal{S}$},
	description={a set of states}
}

\newglossaryentry{setS}
{ 
	name={$\mathcal{S}$},
	description={a set of states}
}

\newglossaryentry{setS}
{ 
	name={$\mathcal{S}$},
	description={a set of states}
}

\newglossaryentry{setS}
{ 
	name={$\mathcal{S}$},
	description={a set of states}
}

\newglossaryentry{setS}
{ 
	name={$\mathcal{S}$},
	description={a set of states}
}

\newglossaryentry{setS}
{ 
	name={$\mathcal{S}$},
	description={a set of states}
}

\newglossaryentry{setS}
{ 
	name={$\mathcal{S}$},
	description={a set of states}
}

\newglossaryentry{setS}
{ 
	name={$\mathcal{S}$},
	description={a set of states}
}

\newglossaryentry{setS}
{ 
	name={$\mathcal{S}$},
	description={a set of states}
}

\newglossaryentry{setS}
{ 
	name={$\mathcal{S}$},
	description={a set of states}
}

\newglossaryentry{setS}
{ 
	name={$\mathcal{S}$},
	description={a set of states}
}

\newglossaryentry{setS}
{ 
	name={$\mathcal{S}$},
	description={a set of states}
}

\newglossaryentry{setS}
{ 
	name={$\mathcal{S}$},
	description={a set of states}
}

\newglossaryentry{setS}
{ 
	name={$\mathcal{S}$},
	description={a set of states}
}

\newglossaryentry{setS}
{ 
	name={$\mathcal{S}$},
	description={a set of states}
}

\newglossaryentry{setS}
{ 
	name={$\mathcal{S}$},
	description={a set of states}
}

\newglossaryentry{setS}
{ 
	name={$\mathcal{S}$},
	description={a set of states}
}

\newglossaryentry{setS}
{ 
	name={$\mathcal{S}$},
	description={a set of states}
}

\newglossaryentry{setS}
{ 
	name={$\mathcal{S}$},
	description={a set of states}
}

\newglossaryentry{setS}
{ 
	name={$\mathcal{S}$},
	description={a set of states}
}

\newglossaryentry{setS}
{ 
	name={$\mathcal{S}$},
	description={a set of states}
}

\newglossaryentry{setS}
{ 
	name={$\mathcal{S}$},
	description={a set of states}
}





\captionsetup[subfloat]{justification=centering}