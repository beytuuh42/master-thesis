\chapter{Introduction}
\label{chap:introduction}
In recent years, research in the field of Artificial Intelligence progressed to the point where Artificial Intelligence systems are able to perform on expert level in their respective fields in complex domains. Specifically, in the strategic games domain. Systems can perform at a world-class level and even defeat humans at that level. 

One of these games is \textit{Go}. Go is a turn-based board game in which the goal is to surround opposing pieces with your own on a (usually) $19 \times 19$ board. Deepmind, an Artificial Intelligence subsidiary of Google, developed a system named \textit{AlphaGo}. AlphaGo was the first system to, play at a world-class level and defeat a world champion in Go alongside various amateur and professional players. Previously, systems were only able to play at amateur level, due to the game its complexity. They were using search trees to play the game. AlphaGo is using search trees as well, however, it is additionally using deep neural networks. Games versus amateur players were utilized to improve gaming by gaining a better grasp of the human gameplay. Also, AlphaGo was training by playing versus itself and learn its own mistakes \cite{silver2017mastering}.

\todo{say neural network}
Another example of a system that is playing on world-class level is \textit{OpenAI Five}. It is a system developed by OpenAI, an Aritifical Intelligence research company. The system is playing the game \textit{Dota 2}. Dota 2 is an online real-time strategy game played by 5 versus 5 players. Each player chooses a character from a pool of 123 heroes, each of which fulfills a specific role in the game. The aim is to destroy a specific opposing structure that is located at the core of the opposing team its base. To achieve this goal, players must collect resources to enhance the abilities and attributes of their heroes and destroy more opposing structures to clear the way to the enemy base. OpenAI Five uses five different agents, each of which controls one of 18 heroes. Similar to AlphaGo, OpenAI Five was trained with various concepts, playing against itself and other human players, such as amateur players from the development team, or online versus amateur players. The level of the amateur players gradually increased, as OpenAI Five got better. The level of those players were measured with the game its own rating system. Gradually, it played against better teams and eventually, OpenAI Five was able to defeat various professional teams, including the world champions of that time \cite{berner2019dota}.

Deepmind developed another Aritifical Intelligence system that is performing at the highest levels in its respective game. The name of this system is \textit{AlphaStar} and its playing the game \textit{Starcraft II}. Starcraft II is another online real-time strategy game. The goal is the same: the destruction of a specific enemy structure in the center of the enemy base. The difference to Dota 2 is, that in Dota 2, the player is controlling one hero at any time, however, in Straftcraft II this is different. The player must control many different units with different purposes at the same time and plays (mostly) against a single player. He has to gather resources to build structures and develop all his units and structures. The structures are responsible for creating units, that will either collect resources or attack the enemy. The system is able to play the game without any restrictions by using deep neural networks. It was able to defeat various amateur players in online matches with varying skill levels based on the game its rating system. Eventually, the system was able to defeat top professional players in matches of five games, without losing once \cite{vinyals2019grandmaster}.

What all these systems have in common is, they are using Reinforcement Learning in their architectures. Reinforcement Learning is a Machine Learning paradigm where an Artificial Intelligence system is learning by interacting with an environment and is rewarded based on its actions, i.e. the system is playing a game either is positively rewarded for an action that has a positive outcome, or in the opposite case, negatively.

This thesis will cover the fundamentals of Reinforcement Learning. Important concepts and technical terms will be explained, as they are necessary in order to comprehend how the presented four methods are working. Custom implementations of the classic board games \textit{Mastermind} and \textit{Battleship} are programmed. The objectives, rules and gameplay of those games will be presented, followed by the implementation and interaction details regarding the games and methods. Finally, experiments will be conducted that will show how the methods perform in these games.



